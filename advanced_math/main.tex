\PassOptionsToPackage{unicode=true}{hyperref} % options for packages loaded elsewhere
\PassOptionsToPackage{hyphens}{url}
%
\documentclass[12pt,a4paper,]{article}
\usepackage{lmodern}
\usepackage{amssymb,amsmath}
\usepackage{ifxetex,ifluatex}
\usepackage{fixltx2e} % provides \textsubscript
\ifnum 0\ifxetex 1\fi\ifluatex 1\fi=0 % if pdftex
  \usepackage[T1]{fontenc}
  \usepackage[utf8]{inputenc}
  \usepackage{textcomp} % provides euro and other symbols
\else % if luatex or xelatex
  \usepackage{unicode-math}
  \defaultfontfeatures{Ligatures=TeX,Scale=MatchLowercase}
  \ifxetex
    \usepackage{xeCJK}
    \setCJKmainfont[]{AR PL KaitiM GB}
  \fi
  \ifluatex
    \usepackage[]{luatexja-fontspec}
    \setmainjfont[]{AR PL KaitiM GB}
  \fi
\fi
% use upquote if available, for straight quotes in verbatim environments
\IfFileExists{upquote.sty}{\usepackage{upquote}}{}
% use microtype if available
\IfFileExists{microtype.sty}{%
\usepackage[]{microtype}
\UseMicrotypeSet[protrusion]{basicmath} % disable protrusion for tt fonts
}{}
\usepackage{hyperref}
\hypersetup{
            pdfborder={0 0 0},
            breaklinks=true}
\urlstyle{same}  % don't use monospace font for urls
\usepackage[right=3cm, left=3cm, top=3.5cm, bottom=3.5cm]{geometry}
\setlength{\emergencystretch}{3em}  % prevent overfull lines
\providecommand{\tightlist}{%
  \setlength{\itemsep}{0pt}\setlength{\parskip}{0pt}}
\setcounter{secnumdepth}{0}
% Redefines (sub)paragraphs to behave more like sections
\ifx\paragraph\undefined\else
\let\oldparagraph\paragraph
\renewcommand{\paragraph}[1]{\oldparagraph{#1}\mbox{}}
\fi
\ifx\subparagraph\undefined\else
\let\oldsubparagraph\subparagraph
\renewcommand{\subparagraph}[1]{\oldsubparagraph{#1}\mbox{}}
\fi

% set default figure placement to htbp
\makeatletter
\def\fps@figure{htbp}
\makeatother

\usepackage{listing}
\lstset{
    basicstyle=\fontsize{10pt}{13pt}\ttfamily\color{Green4!5!black},
    frame=tRBl,
    breakatwhitespace=false,
    keywordstyle=\color{Green4!50!black},
    commentstyle=\color{Gray0!50!black},
    stringstyle=\color{Orange4!50!black},
    breaklines=true,
    xleftmargin=2.5em,
    showstringspaces=false,
}
\usepackage{graphicx}
\usepackage{import}
\usepackage{indentfirst}
\setlength{\parindent}{2em}
\newcommand{\pic}[4]{
    \begin{figure}[htbp]
    \centering
    \includegraphics[width=#1\textwidth]{#2}
    \caption{#3}
    \label{#4}
    \end{figure}
}
\let\paragraph\oldparagraph
\let\subparagraph\oldsubparagraph
\usepackage[bf, big]{titlesec}
\setcounter{secnumdepth}{1}
\titleformat{\section}[frame]
    {\normalfont}
    {\filright\footnotesize\enspace SECTION \thesection\enspace}
    {4pt}{\LARGE\bfseries\filcenter}
\titleformat{\subsection}
    {\normalfont}
    {\thesubsection.}
    {3pt}{\Large\bfseries}
\titleformat{\subsubsection}
    {\normalfont}
    {\thesubsubsection.}
    {2pt}{\large\bfseries}
\titlespacing{\section}{0pt}{1em}{2em}

\date{}

\begin{document}

\hypertarget{ux6570ux5b66ux4e09ux767eux516dux5341ux9898ux8ba1ux5212}{%
\section{数学三百六十题计划}\label{ux6570ux5b66ux4e09ux767eux516dux5341ux9898ux8ba1ux5212}}

\hypertarget{ux8fdbux5ea6ux8868}{%
\subsection{进度表}\label{ux8fdbux5ea6ux8868}}

\begin{verbatim}
[x][x][x][x][x] [x][ ][ ][ ][ ] [ ][ ][ ][ ][ ] [ ][ ][ ][ ][ ]
[ ][ ][ ][ ][ ] [ ][ ][ ][ ][ ] [ ][ ][ ][ ][ ] [ ][ ][ ][ ][ ]
[ ][ ][ ][ ][ ] [ ][ ][ ][ ][ ] [ ][ ][ ][ ][ ] [ ][ ][ ][ ][ ]
[ ][ ][ ][ ][ ] [ ][ ][ ][ ][ ] [ ][ ][ ][ ][ ] [ ][ ][ ][ ][ ]

[ ][ ][ ][ ][ ] [ ][ ][ ][ ][ ] [ ][ ][ ][ ][ ] [ ][ ][ ][ ][ ]
[ ][ ][ ][ ][ ] [ ][ ][ ][ ][ ] [ ][ ][ ][ ][ ] [ ][ ][ ][ ][ ]
[ ][ ][ ][ ][ ] [ ][ ][ ][ ][ ] [ ][ ][ ][ ][ ] [ ][ ][ ][ ][ ]
[ ][ ][ ][ ][ ] [ ][ ][ ][ ][ ] [ ][ ][ ][ ][ ] [ ][ ][ ][ ][ ]

[ ][ ][ ][ ][ ] [ ][ ][ ][ ][ ] [ ][ ][ ][ ][ ] [ ][ ][ ][ ][ ]
[ ][ ][ ][ ][ ] [ ][ ][ ][ ][ ] [ ][ ][ ][ ][ ] [ ][ ][ ][ ][ ]
[ ][ ][ ][ ][ ] [ ][ ][ ][ ][ ] [ ][ ][ ][ ][ ] [ ][ ][ ][ ][ ]
[ ][ ][ ][ ][ ] [ ][ ][ ][ ][ ] [ ][ ][ ][ ][ ] [ ][ ][ ][ ][ ]

[ ][ ][ ][ ][ ] [ ][ ][ ][ ][ ] [ ][ ][ ][ ][ ] [ ][ ][ ][ ][ ]
[ ][ ][ ][ ][ ] [ ][ ][ ][ ][ ] [ ][ ][ ][ ][ ] [ ][ ][ ][ ][ ]
[ ][ ][ ][ ][ ] [ ][ ][ ][ ][ ] [ ][ ][ ][ ][ ] [ ][ ][ ][ ][ ]
[ ][ ][ ][ ][ ] [ ][ ][ ][ ][ ] [ ][ ][ ][ ][ ] [ ][ ][ ][ ][ ]

[ ][ ][ ][ ][ ] [ ][ ][ ][ ][ ]
[ ][ ][ ][ ][ ] [ ][ ][ ][ ][ ]
[ ][ ][ ][ ][ ] [ ][ ][ ][ ][ ]
[ ][ ][ ][ ][ ] [ ][ ][ ][ ][ ]
\end{verbatim}

\hypertarget{ux7ea6ux5b9a}{%
\subsection{约定}\label{ux7ea6ux5b9a}}

\hypertarget{ux7f16ux53f7}{%
\subsubsection{编号}\label{ux7f16ux53f7}}

习题册\texttt{\textless{}book\ number\textgreater{}}使用以下代码表示:

\begin{itemize}
\tightlist
\item
  \texttt{0}:高等数学习题册,高等教育出版社出版,同济大学应用数学系编,
  1996年修订版。
\item
  \texttt{1}:www.mathopolis.com。
\end{itemize}

页码/网页使用\texttt{P:\textless{}page\ number/path\textgreater{}}表示。

顺序号按照\texttt{\textless{}order\ number\textgreater{}}表示,按照第一个完整的题目表示。

注释按照\texttt{\textless{}commant\ text\textgreater{}}表示,按照书上的题号进行表示,使其具备
可读性,更加容易找到原题。

完整格式如下:
\begin{lstlisting}
<book number>-P\[<page number/path>\]-<order number>[(<comment number>)]
\end{lstlisting}

\hypertarget{ux76eeux5f55}{%
\subsubsection{目录}\label{ux76eeux5f55}}

文件夹d表示date,意思是按照日期进行排序的文件。

文件家b表示book,包含了所有的习题册。

\hypertarget{section}{%
\subsection{2020-5-19}\label{section}}

\hypertarget{p192-510.1.5ux9898}{%
\subsubsection{1. 0-P{[}192{]}-5(10.1.5)题}\label{p192-510.1.5ux9898}}

题解见\ref{0-P[192]-5}图。

\pic{0.85}{./d/2020.5.19/00.jpg}{0-P[192]-5(10.1.5)附图}{0-P[192]-5}

\hypertarget{pquestionsq.htmlid6824-1ux9898}{%
\subsubsection{2.
1-P{[}/questions/q.html?id=6824{]}-1题}\label{pquestionsq.htmlid6824-1ux9898}}

\(x^3+C\)。

\hypertarget{pquestionsq.htmlid6825-1ux9898}{%
\subsubsection{3.
1-P{[}/questions/q.html?id=6825{]}-1题}\label{pquestionsq.htmlid6825-1ux9898}}

\(x^5+C\)。

\hypertarget{pquestionsq.htmlid6833-1ux9898}{%
\subsubsection{4.
1-P{[}/questions/q.html?id=6833{]}-1题}\label{pquestionsq.htmlid6833-1ux9898}}

\(3x^30x+C\)。

\hypertarget{pquestionsq.htmlid6842-1ux9898}{%
\subsubsection{5.
1-P{[}/questions/q.html?id=6842{]}-1题}\label{pquestionsq.htmlid6842-1ux9898}}

\(\frac{2}{5}\sqrt{x^5}+C\)。

\hypertarget{pquestionsq.htmlid6846-1ux9898}{%
\subsubsection{6.
1-P{[}/questions/q.html?id=6846{]}-1题}\label{pquestionsq.htmlid6846-1ux9898}}

\((x+1)e^x+C\)。

\hypertarget{pquestionsq.htmlid6849-1ux9898}{%
\subsubsection{7.
1-P{[}/questions/q.html?id=6849{]}-1题}\label{pquestionsq.htmlid6849-1ux9898}}

\((x^2-2x+2)e^x+C\)。

\hypertarget{pquestionsq.htmlid6852-1ux9898}{%
\subsubsection{8.
1-P{[}/questions/q.html?id=6852{]}-1题}\label{pquestionsq.htmlid6852-1ux9898}}

\(\frac{\cos(x)e^x+\sin(x)e^x}{2}\)。

\hypertarget{ux5c0fux7ed3}{%
\subsubsection{小结:}\label{ux5c0fux7ed3}}

复习了一哈哈求积分和有限积分。

积分的话,本质上是求导的逆操作,所以很多定理都是反过来的求导定理。
但是求导简单,求积分不简单。

\begin{itemize}
\item
  简单的求导公式
  $\longrightarrow$求积分表,比如三角,幂函数,log函数,指数函数等
  基本函数。
\item
  (简单的复合函数)函数的加减 $\longrightarrow$简单的积分。
\item
  (不简单的复合函数)函数的积 $\longrightarrow$分部积分。
\item
\end{itemize}

\end{document}
